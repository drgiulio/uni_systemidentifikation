%%%%%%%%%%%%%%%%%%%%%%%%%%%%%%%%%%%%%%%%%
% Short Sectioned Assignment
% LaTeX Template
% Version 1.0 (5/5/12)
%
% This template has been downloaded from:
% http://www.LaTeXTemplates.com
%
% Original author:
% Frits Wenneker (http://www.howtotex.com)
%
% License:
% CC BY-NC-SA 3.0 (http://creativecommons.org/licenses/by-nc-sa/3.0/)
%
%%%%%%%%%%%%%%%%%%%%%%%%%%%%%%%%%%%%%%%%%

%----------------------------------------------------------------------------------------
%	PACKAGES AND OTHER DOCUMENT CONFIGURATIONS
%----------------------------------------------------------------------------------------

\documentclass[paper=a4, fontsize=11pt]{scrartcl} % A4 paper and 11pt font size

\usepackage[T1]{fontenc} % Use 8-bit encoding that has 256 glyphs
%\usepackage{fourier} % Use the Adobe Utopia font for the document - comment this line to return to the LaTeX default
\usepackage[english]{babel} % English language/hyphenation
\usepackage{amsmath,amsfonts,amsthm} % Math packages

\usepackage{lipsum} % Used for inserting dummy 'Lorem ipsum' text into the template

\usepackage{sectsty} % Allows customizing section commands
\allsectionsfont{\centering \normalfont\scshape} % Make all sections centered, the default font and small caps

\usepackage{fancyhdr} % Custom headers and footers
\pagestyle{fancyplain} % Makes all pages in the document conform to the custom headers and footers
\fancyhead{} % No page header - if you want one, create it in the same way as the footers below
\fancyfoot[L]{} % Empty left footer
\fancyfoot[C]{} % Empty center footer
\fancyfoot[R]{\thepage} % Page numbering for right footer
\renewcommand{\headrulewidth}{0pt} % Remove header underlines
\renewcommand{\footrulewidth}{0pt} % Remove footer underlines
\setlength{\headheight}{13.6pt} % Customize the height of the header

\numberwithin{equation}{section} % Number equations within sections (i.e. 1.1, 1.2, 2.1, 2.2 instead of 1, 2, 3, 4)
\numberwithin{figure}{section} % Number figures within sections (i.e. 1.1, 1.2, 2.1, 2.2 instead of 1, 2, 3, 4)
\numberwithin{table}{section} % Number tables within sections (i.e. 1.1, 1.2, 2.1, 2.2 instead of 1, 2, 3, 4)

\setlength\parindent{0pt} % Removes all indentation from paragraphs - comment this line for an assignment with lots of text

%----------------------------------------------------------------------------------------
%	TITLE SECTION
%----------------------------------------------------------------------------------------

\newcommand{\horrule}[1]{\rule{\linewidth}{#1}} % Create horizontal rule command with 1 argument of height

\title{	
\normalfont \normalsize 
\textsc{Technische Universit�t M�nchen, Lehrstuhl f�r Elektrische Antriebe und Leistungselektronik} \\ [25pt] % Your university, school and/or department name(s)
\horrule{0.5pt} \\[0.4cm] % Thin top horizontal rule
\huge Projekt: Identifikation eines Duffing Systems mit einem Neuronalen Netz \\ % The assignment title
\horrule{2pt} \\[0.5cm] % Thick bottom horizontal rule
}

\author{Maximilian Schermer, Matrikelnummer: 03664650, \\ Maximilian Sperr, Matrikelnummer: 03658841,\\ Giulio Evangelisti,  Matrikelnummer: 03659301} % Your name

\date{\normalsize\today} % Today's date or a custom date

\begin{document}

\maketitle % Print the title

%----------------------------------------------------------------------------------------
%	PROBLEM 1
%----------------------------------------------------------------------------------------

\section{Duffing System}

System beschreiben, Simulationsergebnisse des Duffing Systems zeigen.

%------------------------------------------------

\section{Identifikationsmodell}

Als Identifikationsmodell dient das General Dynamic Neuronal Network (GDNN) welches aus drei versteckten Schichten mit zweimal zwei und einmal einem Neuron (2-2-1) besteht. In der Eingangs sowie in der Ausgangsschicht befindet sich ein Neuron und die verschiedenen Schichten sind miteinander �ber Tapped Delay Lines gekoppelt. Die Tapped Delay Lines sind wiefolgt aufgebaut:

\begin{table}
	\centering
		\begin{tabular}{*{3}{p{4cm}}}
			Schicht 1 & Schicht 2 & Schicht 3 \\
			$DI^{1,1} =  \left\{1,2,3\right\}$ & $DL^{2,1} =  \left\{0\right\}$ & $DL^{3,2} =  \left\{0\right\}$ \\
			$DL^{1,1} =  \left\{1,2,3\right\}$ & $DL^{2,2} =  \left\{1,2,3\right\}$ & $DL^{3,3} =  \left\{1,2,3\right\}$ \\
			$DL^{1,2} =  \left\{1,2,3\right\}$ & $DL^{2,3} =  \left\{1,2,3\right\}$ \\
			$DL^{1,3} =  \left\{1,2,3\right\}$ \\			
		\end{tabular}
\end{table}

Die Identifikation findet mittels eines NARX Modells statt, d.h. der Systemausgang und das Anregungssignal sind die Eing�nge f�r das neuronale Netz. Das GDNN wurde wegen seiner hohen Approximationsf�higkeit ausgew�hlt. 
%------------------------------------------------

\section{Systemanregung}

Anregungssignal beschrieben, Grund f�r Chirp angeben, Frequenzabh�ngigkeit des Systems.


%------------------------------------------------

\section{Identifikationsergebnisse}

Identifikationsergebnis, Validierung Modell


%------------------------------------------------

\section{Anpassung des Identifikationsmodells}

Modellstruktur und Anregungssignal �ndern. Zu statischen Identifikationsmodell wechseln.




\end{document}